\documentclass[12pt, a4paper]{article}

\usepackage{amsmath}
\usepackage{amssymb}
\usepackage{enumerate}

\setlength\parskip{1em}
\setlength\parindent{0em}

\title{Assignment 1}

\author{Hendrik Werner s4549775}

\begin{document}
\maketitle

\section{} %1
\begin{enumerate}[a]
	\item %a
	\item %b
	\item %c
	\item %d
	\item %e
	\item %f
	\item %g
	\item %h
\end{enumerate}

\section{} %2
\begin{enumerate}[a]
	\item %a
	\item %b
	\item %c
	\item %d
\end{enumerate}

\section{} %3
\begin{enumerate}[a]
	\item %a
	$\square ((A_{S0} \rightarrow \lnot A_{S1}) \land (A_{S1} \rightarrow \lnot A_{S0}))$
	\item %b
	$\square (A_{S0} \lor A_{S1})$
	\item %c
	$\square \diamond (A_{S0} \lor A_{S1})$
	\item %d
	$\square ((A_{S0} \rightarrow \diamond A_{R0}) \land (A_{S1} \rightarrow \diamond A_{R1}))$
	\item %e
	$\square ((A_{S0} \rightarrow \bigcirc B_{R0}) \land (A_{S1} \rightarrow \bigcirc B_{R1}))$
	\item %f
	$\square (B_{R0} \rightarrow (B_{S0} \cup B_{R1}))$
	\item %g
	$\square (A_{S0} \rightarrow \bigcirc (A_{S0} \cup A_{R0}))$
	\item %h
	$\square \bigcirc (A_{S0} \lor A_{S1} \lor A_{R0} \lor A_{R1})$

	I am not sure if this is useful, because $\square \bigcirc (A_{S0} \lor A_{S1} \lor A_{R0} \lor A_{R1}) = \square \bigcirc True = True$ so it does not really do anything. But on the other hand that is also kind of what the rule "A can always send." does: nothing. This sentence is a NOP so is the formula.
	\item %i
\end{enumerate}

\end{document}
