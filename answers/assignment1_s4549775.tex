\documentclass[12pt, a4paper]{article}

\usepackage{amsmath}
\usepackage{amssymb}
\usepackage{enumerate}

\setlength\parskip{1em}
\setlength\parindent{0em}

\title{Assignment 1}

\author{Hendrik Werner s4549775}

\begin{document}
\maketitle

\section{} %1
\begin{enumerate}[a]
	\item %a
	Holds because every node in the graph has a path to $\{a\}$.
	\item %b
	Does not hold because 3, 4, and 5 are a SCC (strongly connected component) and $\{a\} \not \in 3$.
	\item %c
	Does not hold. Conterexample: $\varnothing \{b\} \{a, c\} \{a, c\} \{a, c\} \dots$
	\item %d
	Holds because at 1 we have two choices: 2, 3. If we take 2 the property holds trivially. If we chose 3 the property also holds trivially.
	\item %e
	Does not hold. Conterexample: $\varnothing \{b\} \{b\} \{b\} \dots$
	\item %f
	\item %g
	\item %h
\end{enumerate}

\section{} %2
\begin{enumerate}[a]
	\item %a
	$\diamond \square \phi \neq \square \diamond \phi$ because $\{\phi\} \varnothing \{\phi\} \varnothing \{\phi\} \varnothing \dots$ satisfies $\square \diamond \phi$ but not $\diamond \square \phi$.
	\item %b
	$\bigcirc \diamond \phi = \diamond \bigcirc \phi$ because

	$\sigma \models \bigcirc \diamond \phi\\
	\Leftrightarrow \sigma[1..] \models \diamond \phi\\
	\Leftrightarrow \exists j \geq 0 (\sigma[1..][j..] \models \phi)\\
	\Leftrightarrow \exists j \geq 0 (\sigma[j+1..] \models \phi)\\
	\Leftrightarrow \exists j \geq 0 (\sigma[j..][1..] \models \phi)\\
	\Leftrightarrow \exists j \geq 0 (\sigma[j..] \models \bigcirc \phi)\\
	\Leftrightarrow \sigma \models \diamond \bigcirc \phi$
	\item %c
	$\diamond \phi \lor \diamond \psi = \diamond (\phi \lor \psi)$ because

	$\sigma \models \diamond \phi \lor \diamond \psi\\
	\Leftrightarrow \exists j \geq 0 (\sigma[j..] \models \phi) \lor \exists j \geq 0 (\sigma[j..] \models \psi)\\
	\Leftrightarrow \exists j \geq 0 (\sigma[j..] \models \phi \lor \psi)\\
	\Leftrightarrow \sigma \models \diamond (\phi \lor \psi)$
	\item %d
	$\diamond \phi \land \diamond \psi \neq \diamond (\phi \land \psi)$ because $\{\phi\} \{\phi\} \{\phi\} \{\psi\} \{\psi\} \{\psi\}$ satisfies $\diamond \phi \land \diamond \psi$ but not $\diamond (\phi \land \psi)$.
\end{enumerate}

\section{} %3
\begin{enumerate}[a]
	\item %a
	$\square ((A_{S0} \rightarrow \lnot A_{S1}) \land (A_{S1} \rightarrow \lnot A_{S0}))$
	\item %b
	$\square (A_{S0} \lor A_{S1})$
	\item %c
	$\square \diamond (A_{S0} \lor A_{S1})$
	\item %d
	$\square ((A_{S0} \rightarrow \diamond A_{R0}) \land (A_{S1} \rightarrow \diamond A_{R1}))$
	\item %e
	$\square ((A_{S0} \rightarrow \bigcirc B_{R0}) \land (A_{S1} \rightarrow \bigcirc B_{R1}))$
	\item %f
	$\square (B_{R0} \rightarrow (B_{S0} \cup B_{R1}))$
	\item %g
	$\square (A_{S0} \rightarrow \bigcirc (A_{S0} \cup A_{R0}))$
	\item %h
	$\square \bigcirc (A_{S0} \lor A_{S1} \lor A_{R0} \lor A_{R1})$

	I am not sure if this is useful, because $\square \bigcirc (A_{S0} \lor A_{S1} \lor A_{R0} \lor A_{R1}) = \square \bigcirc True = True$ so it does not really do anything. But on the other hand that is also kind of what the rule "A can always send." does: nothing. This sentence is a NOP so is the formula.
	\item %i
	$\square ((A_{S0} \rightarrow (\lnot B_{R0} \lor B_{R0})) \land (A_{S1} \rightarrow (\lnot B_{R1} \lor B_{R1})))$

	This has the same problem: $\square ((A_{S0} \rightarrow (\lnot B_{R0} \lor B_{R0})) \land (A_{S1} \rightarrow (\lnot B_{R1} \lor B_{R1}))) = \square ((A_{S0} \rightarrow True) \land (A_{S1} \rightarrow True)) = \square (True \land True) = \square True = True$ so it does not really do anything. On the other hand as with (h) this is also true for the sentence this was derived from. "if A sends, B may or may not receive" has no declarative value so it matched the formula. The sentence has the form $A \rightarrow B \lor \lnot B = A \rightarrow True = True$.

	To me it makes sense to derive NOP formulas from NOP sentences.
\end{enumerate}

\end{document}
